\chapter[Implementacja][Implementacja]{Implementacja}
Podczas prac nad systemem powstały takie artefakty jak:
\begin{itemize}
  \item skrypt uruchamiający usługę,
  \item aplikacja kliencka,
  \item aplikacja serwerowa,
  \item skrypt demostracyjny.
\end{itemize}

\section[Interfejs usługi][Interfejs usługi]{Interfejs usługi}
W~ramach prac nad~systemem opracowano API zdefiniowane w~języku Apache
Thrift~IDL, które przedstawiono na~poniższym wydruku:

\begin{lstlisting}[language=C,style=incode,
morekeywords={exception,typedef,i32,i64,namespace,enum,required,struct,bool}]
namespace java rso.dfs.generated

typedef i32 int
typedef i64 long
typedef string IPType

enum ServerType {
	Slave,
	Master,
	Shadow
}

struct ServerStatus
{
	1: required ServerType type;
	2: required i32 filesNumber;
	3: required i64 freeSpace;
	4: required i64 usedSpace;
	5: required IPType serverIP;
}

struct SystemStatus {
	1:required i32 filesNumber;
	2:required list<ServerStatus> serversStatuses;
}

struct CoreStatus {
	1:IPType masterAddress;
	2:list<IPType> shadowsAddresses;
}

// describes part of file
struct FilePartDescription {
	1:int fileId;
	2:long offset;
}

//represents part of a file
struct FilePart {
	1:int fileId;
	2:long offset;
	3:binary data;
}

// communication initiated by new node
// new node sends to master file list
struct NewSlaveRequest {
	1:required IPType slaveIP;
	2:list<int> fileIds;
	3:list<long> fileSizes;
}
struct GetFileParams
{
	1:required i32 fileId;
	2:required IPType slaveIp;
	3:required i64 size;
}

struct PutFileParams
{
	1:required bool canPut;
	2:required i32 fileId;
	3:required IPType slaveIp;
}

service Service
{
	//returns administrative system status
	SystemStatus getStatus(),
	
	// returns list of file names 
	list<string> listFileNames(),
	
	//infrastucure building
	//slave sends request to master to register to serve
	CoreStatus registerSlave(1:NewSlaveRequest req),
	
	//with this request master makes slave register again.
	void forceRegister(1:CoreStatus status),
	
	//master updates slaves status
	void updateCoreStatus(1:CoreStatus status),
	
	//master sends request to slave
	void becomeShadow(1: CoreStatus status),
	
	//client - anyone
	CoreStatus getCoreStatus(), // returns master/shadows list
	
	//ping server for checking whether it's alive
	void pingServer(),
	
	// file id, file size; master slave
	// force slave to be ready for file (fileId)
	// which will be sent from client
	void prepareForReceiving(1: int fileID, 2:long size),
	//file id, slave ip
	void replicate(1:int fileID, 2:IPType slaveIP, 3:long size),
	
	//master - slave
	bool isFileUsed(1:int fileID),
	void removeFileSlave(1:int fileID),
	
	GetFileParams getFile(1:string filepath),
	GetFileParams getFileFailure(1:string filepath),
	PutFileParams putFile(1:string filepath, 2:long size),
	PutFileParams putFileFailure(1:string filepath, 2:long size),
	bool removeFile(1:string filepath),
	
	// returns: which part of file should be sent next
	FilePartDescription sendFileToSlaveRequest(1: int fileId),
	
	// Input: part of file which has to be sent
	// returns: which part of file should be sent next,
	//          special value in case of finish
	FilePartDescription sendFilePartToSlave(1: FilePart filePart),
	FilePart getFileFromSlave(1: FilePartDescription filePartDescription),
	
	void fileUploadSuccess(1:int fileID, 2: IPType slaveIP) 
}
\end{lstlisting}

\section[Zarządzanie usługą][Zarządzanie usługą]{Zarządzanie usługą}
Podczas prac nad~systemem wytworzono skrypt powłoki (\emph{shell script})
ułatwiający zarządzanie usługą. Umożliwa on~m.in.~uruchamianie oraz
zatrzymywanie usługi, a~także pozwala wyświetlić informacje o~bieżącym stanie
usługi. W~celu korzystania ze~skryptu konieczne jest posiadanie systemu
z~rodziny Unix z~zainstalowanymi narzędziami: \emph{awk, timeout oraz nmap}.
Możliwe wywołania skryptu zamieszczono na~poniższym zestawieniu.

\vspace{5mm}
\renewcommand{\arraystretch}{1.5}
\begin{tabular}[h]{p{3cm} p{11cm}}
\emph{start}	[timeout] & próbuje uruchomić \emph{naming service} na~każdym z~IP,
po~pierwszym sukcesie uruchamia slave'y. Implementacja daje [timeout
w~milisekundach] milisekund (domyślnie 5000) na~uruchomienie się~aplikacji serwera od~chwili wykonania komendy ssh. \\
\emph{stop} & zatrzymuje usługę \\
haltvm & zatrzymuje każdy z~serwerów (wykonuje na~nim polecenie powłoki
\emph{halt}) \\
\emph{status} & uruchamia \emph{system status} dla~pierwszego~servera, z~którym
uda się~zkomunikować \\
\emph{client} & nawiązuje połączenie z~pierwszym serwerem, z~którym uda
się~zkomunikować \\
\end{tabular}
