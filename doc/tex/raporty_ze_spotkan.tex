\chapter[Raporty ze spotkań][Raporty ze spotkań]{Raporty ze spotkań}
\section[Spotkanie nr 1][Spotkanie nr 1]{Spotkanie nr 1}

\noindent
\textbf{data}: 2014-03-04 \\
\textbf{czas}: 18:00-18:30

\vspace{5mm}
\noindent
\textbf{uczestnicy}:
\begin{itemize}
	\item Jacek Witkowski
	\item Mateusz Statkiewicz
	\item Marcin Swend
	\item Piotr Kalinowski
\end{itemize}

\vspace{5mm}
\noindent
\textbf{cele}:
\begin{itemize}
  \item Organizacja:
  \begin{itemize}
    \item wymiana danych kontaktowych,
    \item określenie swoich preferencji co~do~ról istniejących w~projekcie,
    \item ustalenie możliwych terminów i formy spotkań.
  \end{itemize}
  \item Wstępna analiza wymagań dotyczących realizacji projektu.
  \item Przegląd i~ewentualny wybór narzędzi używanych podczas realizacji
  projektu.
\end{itemize}

\vspace{5mm}
\noindent
\textbf{ustalenia}: \\
Każdy z członków zespołu określił role projektowe, które chciałby wykonywać.
Wstępnie przyjęto następujący podział ról:
\begin{itemize}
  \item architekt: Marcin Swend,
  \item zarządca repozytorium: Jacek Witkowski,
  \item dokumentalista: Jacek Witkowski,
  \item tester: Mateusz Statkiewicz,
  \item handlowiec: Jacek Witkowski.
\end{itemize}

\vspace{5mm}
Ponadto, każdy z członków zespołu (poza kierownikiem projektu) przyjmuje rolę
programisty.

\vspace{5mm}
Ustalono, że możliwe terminy spotkań to wtorki o godzinie 18.00 oraz piątki
o godzinie 18.00. Istnieje również możliwość odbywania telekonferencji
w weekendy.

\vspace{5mm}
Dokumentacja będzie prowadzona przy użyciu Google Docs, ponieważ każdy
z członków projektu już teraz posiada konto w serwisie Google. Stąd, korzystanie
z narzędzia nie niesie ze sobą konieczności dodatkowego rejestrowania się.
Do~zarządzania projektem zostanie użyty system Redmine, który zostanie
uruchomiony na serwerze Michała Statkiewicza. Jako narzędzie kontroli wersji
zostanie wykorzystany git. Repozytorium zostanie utworzone w serwisie GitHub.

\section[Spotkanie nr 2][Spotkanie nr 2]{Spotkanie nr 2}

\noindent
\textbf{data}: 2014-03-11 \\
\textbf{czas}: 18:00-18:30

\vspace{5mm}
\noindent
\textbf{uczestnicy}:
\begin{itemize}
	\item Jacek Witkowski
	\item Mateusz Statkiewicz
	\item Marcin Swend
	\item Adam Papros
	\item Marcin Dzieżyc
\end{itemize}

\vspace{5mm}
\noindent
\textbf{cele}:
\begin{itemize}
  \item ustalenie wymagań funkcjonalnych projektu,
  \item ustalenie założeń dotyczących środowiska uruchomieniowego,
  \item ustalenie zawartości pliku konfiguracyjnego,
  \item ustalenie wstępnej architektury systemu.
\end{itemize}

\vspace{5mm}
\noindent
\textbf{ustalenia}: \\
Założenia dotyczące projektu:
\begin{itemize}
	\item aplikacja kliencka będzie programem konsolowym,
	\item będą możliwe do użycia następujące komendy:
	\begin{itemize}
		\item ls - wylistowanie zawartości folderu,
		\item cd - zmiana bieżącego katalogu,
		\item rm - usunięcie wskazanego pliku lub folderu,
		\item cp - kopiowanie pliku/folderu z~miejsca źródłowego do~docelowego,
		\item mv - przemieszczenie pliku/folderu z miejsca źródłowego do~docelowego.
   \end{itemize}
\end{itemize}

\vspace{5mm}
Środowisko uruchomieniowe: 4 laptopy działające w~jednej podsieci. 
Laptopy będą połączone za~pomocą routera oferującego możliwość
bezprzewodowego połączenia. Na~laptopach uruchomione będą maszyny wirtualne,
na~których uruchomione będą aplikacje klienckie lub serwerowe.

\vspace{5mm}
\noindent
Spis rzeczy, które muszą znaleźć się w pliku konfiguracyjnym:
\begin{itemize}
  \item lista reprezentująca adresy ip dozwolone dla urządzeń stanowiących
  część serwerową w systemie,
  \item stopień redundancji danych.
\end{itemize}

\vspace{5mm}
\noindent
Wstępna architektura systemu:
\begin{itemize}
  \item urządzenia stanowiące część usługową będą podzielone na dwie klasy:
część lokalizacyjno-nazewniczą oraz cześć przechowawczą,
  \item wśród urządzeń klasy lokalizacyjno-nazewniczej wyróżnione będzie jedno
  urządzenie typu master, które będzie przechowywać informacje o~rozmieszczeniu
  plików w~katalogach oraz o~ich~lokalizacji na~poszczególnych urządzeniach
  przechowawczych.
  \item urządzenia z~klasy lokalizacyjno-nazewniczej nie~będące masterem, będą
  tzw.~\emph{shadow masterami} stanowiącymi dokładną kopię mastera i~będącymi 
  w~stanie go~zastąpić w~razie jego awarii.
\end{itemize}

\section[Spotkanie nr 3][Spotkanie nr 3]{Spotkanie nr 3}

\noindent
\textbf{data}: 2014-03-14 \\
\textbf{czas}: 16:00-18:30

\vspace{5mm}
\noindent
\textbf{uczestnicy}:
\begin{itemize}
	\item Marcin Dzieżyc
	\item Piotr Kalinowski
	\item Adam Papros
	\item Mateusz Statkiewicz
	\item Marcin Swend
	\item Jacek Witkowski
\end{itemize}

\vspace{5mm}
\noindent
\textbf{cele}: \\
Ustalenie scenariuszy dla następujących działań:
\begin{itemize}
  \item wysyłanie pliku,
  \item pobieranie pliku.
\end{itemize}

\vspace{5mm}
\noindent
\textbf{ustalenia}: \\
Scenariusz pobierania pliku:
\begin{enumerate}
	\item Klient łączy się do dowolnego serwera.
	\item Klient otrzymuje adres mastera.
	\item Klient łączy się z masterem.
	\item Klient wydaje polecenie pobrania pliku o określonej ścieżce.
	\item Master wyszukuje plik w hierarchii plików.
	\item Master odsyła do klienta listę slave'ów przechowujących pliki wraz
	z~identyfikatorem pliku.
	\item Klient nawiązuje połączenie z~pierwszym z~listy slave'ów i~wysyła żądanie
	pobrania pliku o~wskazanym identyfikatorze od~zerowego bajta.
	\item Slave rozpoczyna wysyłanie pliku.
	\item Klient odbiera plik.
\end{enumerate}

\vspace{5mm}
Scenariusz wysyłania pliku:
\begin{enumerate}
	\item Klient łączy się do dowolnego serwera.
	\item Klient otrzymuje adres mastera.
	\item Klient łączy się z masterem.
	\item Klient informuje mastera o~chęci wysłania pliku o~określonej ścieżce
	i~rozmiarze.
	\item Master sprawdza czy plik już~istnieje oraz wysyła do~wybranych slave'ów
	informację o~identyfikatorze pliku, który ma~zostać odebrany wraz z~listą
	wszystkich slave'ów, do~których została wysłana ta~lista (każdy z~wybranej
	grupy slave'ów otrzymuje tę~samą listę).
	\item Master wysyła identyfikator pliku i~listę która była rozsyłana w~p.~5
	do~klienta.
	\item Klient łączy się z pierwszym slave'em z listy i~podaje identyfikator
	pliku.
	\item Klient rozpoczyna wysyłanie pliku do slave'a.
	\item Slave odbiera fragmenty pliku i~przesyła je~do~kolejnego slave'a
	z~otrzymanej od~mastera listy. Slave'y rozsyłają między sobą odbierane fragmenty.
	\item Klient kończy wysyłanie.
\end{enumerate}

\section[Spotkanie nr 4][Spotkanie nr 4]{Spotkanie nr 4}

\noindent
\textbf{data}: 2014-03-18 \\
\textbf{czas}: 18:00-21:15

\vspace{5mm}
\noindent
\textbf{uczestnicy}:
\begin{itemize}
	\item Marcin Dzieżyc
	\item Piotr Kalinowski
	\item Adam Papros
	\item Mateusz Statkiewicz
	\item Marcin Swend
	\item Jacek Witkowski
\end{itemize}

\vspace{5mm}
\noindent
\textbf{cele}: \\
Ustalenie scenariuszy dla następujących działań:
\begin{itemize}
  \item uruchomienie systemu,
  \item usuwanie pliku z systemu,
  \item kopiowanie pliku do systemu,
  \item kopiowanie pliku z systemu,
  \item otrzymanie informacji o stanie systemu (jako całości oraz poszczególnych
  węzłów).
\end{itemize}

\vspace{5mm}
\noindent
\textbf{ustalenia}: \\
W warstwie lokalizacyjno-nazewniczej istnieje tylko jedno aktywne urządzenie
obsługujące wszystkie żądania trafiające do tej warstwy. Urządzenie to będziemy
nazywać masterem. Pozostałe urządzenia z tej warstwy będą jedynie kopiami
mastera. Będziemy je nazywać Shadow Masterami.

\vspace{5mm}
Master oraz Shadow Mastery wykorzystują synchronizację on-line. Utrzymują
lokalnie tablicę trwających procesów, która jest między nimi synchronizowana
w~ten sposób, by~awaria mastera w~dowolnym momencie nie spowodowała awarii
całego systemu.

\vspace{5mm}
Warunkiem koniecznym uznania dowolnego żądania klienta za zakończone, jest
zsynchronizowanie informacji dotyczących tego żądania na masterze oraz shadow
masterach.

\vspace{5mm}
W pliku konfiguracyjnym będzie zawarta lista urządzeń, które mają być
pierwotnymi urządzeniami istniejącymi w systemie. Informacja ta będzie
wykorzystywana w skrypcie uruchamiającym cały system. Zakładamy, że podczas
działania skryptu urządzenie, które udało się uruchomić nie ulegnie awarii.

\vspace{5mm}
Na spotkaniu opracowano również kolejne wersję scenariuszy wysyłania pliku
do~systemu oraz jego odbierania, a także usuwania. Scenariusze te zostały
umieszczone w dokumentacji projektu.

\section[Spotkanie nr 5][Spotkanie nr 5]{Spotkanie nr 5}

\noindent
\textbf{data}: 2014-03-21 \\
\textbf{czas}: 16:00-16:30

\vspace{5mm}
\noindent
\textbf{uczestnicy}:
\begin{itemize}
	\item Marcin Dzieżyc
	\item Piotr Kalinowski
	\item Mateusz Statkiewicz
	\item Jacek Witkowski
\end{itemize}

\vspace{5mm}
\noindent
\textbf{cele}: \\
Weryfikacja scenariuszy opracowanych przez zespół dla następujących czynności:
\begin{itemize}
  \item wysyłanie pliku do systemu,
  \item pobieranie pliku z systemu,
  \item usuwanie pliku z systemu.
\end{itemize}

\vspace{5mm}
Zaplanowanie czynności do wykonania w~ciągu najbliższych 2 tygodni.

\vspace{5mm}
\noindent
\textbf{ustalenia}: \\
W ciągu najbliższych dwóch tygodni należy:
\begin{itemize}
  \item ustalić metody udostępniane przez aplikację serwerową oraz aplikację kliencką,
  \item ustalić struktury komunikatów wymienianych między urządzeniami
  działającymi w systemie (komuniakcja wewnątrzsystemowa oraz komunikacja z klientami).
\end{itemize}

\section[Spotkanie nr 6][Spotkanie nr 6]{Spotkanie nr 6}

\noindent
\textbf{data}: 2014-03-25 \\
\textbf{czas}: 16:00-16:30

\vspace{5mm}
\noindent
\textbf{uczestnicy}:
\begin{itemize}
	\item Marcin Dzieżyc
	\item Piotr Kalinowski
	\item Adam Papros
	\item Mateusz Statkiewicz
	\item Marcin Swend
	\item Jacek Witkowski
\end{itemize}

\vspace{5mm}
\noindent
\textbf{cele}: \\
Ustalenie artefaktów jakie mają powstać w drugim etapie projektu. Podział zadań
dotyczących implementacji podstawowych funkcjonalności.

\vspace{5mm}
\noindent
\textbf{ustalenia}: \\
Ustalono, że~w~ramach drugiego etapu wykonany zostanie system realizujący
funkcjonalności: wysyłania/pobierania danych do/z~systemu, usuwania plików,
pobierania informacji o~urządzeniach działających w~systemie (operacja status),
uruchamianie/zatrzymywanie systemu, przyłączanie nowych urządzeń do~systemu.

\vspace{5mm}
W pierwszej kolejności należy ustalić interfejsy jakie mają oferować aplikacje
klienta, mastera oraz slave'a oraz komunikaty wymieniane między nimi.
Podzielono zadania w następujący sposób:
\begin{itemize}
	\item komunikacja Master-Klient: Piotr Kalinowski, Jacek Witkowski,
	\item komunikacja Master-Slave: Marcin Dzieżyc, Adam Papros,
	\item komunikacja Klient-Slave: Mateusz Statkiewicz, Marcin Swend.
\end{itemize}

\vspace{5mm}
Zadania należy wykonać do 4~kwietnia.


\section[Spotkanie nr 7][Spotkanie nr 7]{Spotkanie nr 7}

\noindent
\textbf{data}: 2014-04-04 \\
\textbf{czas}: 16:00-16:30

\vspace{5mm}
\noindent
\textbf{uczestnicy}:
\begin{itemize}
	\item Marcin Dzieżyc
	\item Piotr Kalinowski
	\item Marcin Swend
	\item Jacek Witkowski
\end{itemize}

\vspace{5mm}
\noindent
\textbf{cele}: \\
Zebranie i uspójnienie przygotowanych interfejsów i wiadomości koniecznych
do komunikacji pomiędzy klientem, masterem i slave'em.

\vspace{5mm}
\noindent
\textbf{ustalenia}: \\
Po ujednoliceniu stworzonych interfejsów oraz postaci komunikatów
przydzielono zadania implementacji poszczególnych funkcjonalności
członków zespołu.

\section[Spotkanie nr 8][Spotkanie nr 8]{Spotkanie nr 8}

\noindent
\textbf{data}: 2014-03-18 \\
\textbf{czas}: 16.00-17.00

\vspace{5mm}
\noindent
\textbf{uczestnicy}:
\begin{itemize}
	\item Marcin Dzieżyc
	\item Piotr Kalinowski
	\item Mateusz Statkiewicz
	\item Marcin Swend
	\item Jacek Witkowski
\end{itemize}

\vspace{5mm}
\noindent
\textbf{cele}: \\
Dopracowanie serwisów jakie mają powstać w~systemie oraz wstępne ustalenie
podziału zadań pomiędzy członkami zespołu.

\section[Spotkanie nr 9][Spotkanie nr 9]{Spotkanie nr 9}

\noindent
\textbf{data}: 2014-04-22 \\
\textbf{czas}: 18.00-19.30

\vspace{5mm}
\noindent
\textbf{uczestnicy}:
\begin{itemize}
	\item Marcin Dzieżyc
	\item Piotr Kalinowski
	\item Adam Papros
	\item Mateusz Statkiewicz
	\item Marcin Swend
	\item Jacek Witkowski
\end{itemize}

\vspace{5mm}
\noindent
\textbf{cele}: \\
Ustalenie postępu prac. Identyfikacja problemów jakie powstały podczas
implementowania serwisów oraz próba ich~rozwiązania.

\section[Spotkanie nr 10][Spotkanie nr 10]{Spotkanie nr 10}

\noindent
\textbf{data}: 2014-05-05 \\
\textbf{czas}: 19.00-22.00

\vspace{5mm}
\noindent
\textbf{uczestnicy}:
\begin{itemize}
	\item Marcin Dzieżyc
	\item Piotr Kalinowski
	\item Adam Papros
	\item Mateusz Statkiewicz
	\item Marcin Swend
\end{itemize}

\vspace{5mm}
\noindent
\textbf{cele}: \\
Finalizacja prac nad~implementacją funkcjonalności zaplanowanych do~wykonania
w~ramach drugiej fazy projektu.

\section[Spotkanie nr 11][Spotkanie nr 11]{Spotkanie nr 11}

\noindent
\textbf{data}: 2014-13-05 \\
\textbf{czas}: 18.00-19.00

\vspace{5mm}
\noindent
\textbf{uczestnicy}:
\begin{itemize}
	\item Marcin Dzieżyc
	\item Piotr Kalinowski
	\item Adam Papros
	\item Mateusz Statkiewicz
	\item Marcin Swend
	\item Jacek Witkowski
\end{itemize}

\vspace{5mm}
\noindent
\textbf{cele}: \\
Określenie postępu prac nad~implementacją replikacji plików.

\section[Spotkanie nr 12][Spotkanie nr 12]{Spotkanie nr 12}

\noindent
\textbf{data}: 2014-05-20 \\
\textbf{czas}: 18.00-19.00

\vspace{5mm}
\noindent
\textbf{uczestnicy}:
\begin{itemize}
	\item Marcin Dzieżyc
	\item Piotr Kalinowski
	\item Mateusz Statkiewicz
	\item Marcin Swend
	\item Jacek Witkowski
\end{itemize}

\vspace{5mm}
\noindent
\textbf{cele}: \\
Określenie postępu prac nad~implementacją replikacji plików. Przydział zadań
implementacji obsługi sytuacji awaryjnych występujących w~systemie.

\section[Spotkanie nr 13][Spotkanie nr 13]{Spotkanie nr 13}

\noindent
\textbf{data}: 2014-05-23 \\
\textbf{czas}: 16.00-17.00

\vspace{5mm}
\noindent
\textbf{uczestnicy}:
\begin{itemize}
	\item Marcin Dzieżyc
	\item Piotr Kalinowski
	\item Mateusz Statkiewicz
	\item Jacek Witkowski
\end{itemize}

\vspace{5mm}
\noindent
\textbf{cele}: \\
Ustalenie postępu prac nad~projektem. Identyfikacja powstałych problemów oraz
próba ich~rozwiązania.

\section[Spotkanie nr 14][Spotkanie nr 14]{Spotkanie nr 14}

\noindent
\textbf{data}: 2014-05-24 \\
\textbf{czas}: 9.00-13.00

\vspace{5mm}
\noindent
\textbf{uczestnicy}:
\begin{itemize}
	\item Marcin Dzieżyc
	\item Piotr Kalinowski
	\item Mateusz Statkiewicz
	\item Marcin Swend
	\item Jacek Witkowski
\end{itemize}

\vspace{5mm}
\noindent
\textbf{cele}: \\
Ustalenie postępu prac nad~projektem. Wspólna implementacja obsługi sytuacji
błędnych. Dopracowywanie replikacji plików oraz masterów.

\section[Spotkanie nr 15][Spotkanie nr 15]{Spotkanie nr 15}

\noindent
\textbf{data}: 2014-06-01 \\
\textbf{czas}: 18.30-22.00

\vspace{5mm}
\noindent
\textbf{uczestnicy}:
\begin{itemize}
	\item Marcin Dzieżyc
	\item Piotr Kalinowski
	\item Mateusz Statkiewicz
	\item Marcin Swend
	\item Jacek Witkowski
\end{itemize}

\vspace{5mm}
\noindent
\textbf{cele}: \\
Przeprowadzenie pełnych testów aplikacji.

\section[Spotkanie nr 16][Spotkanie nr 16]{Spotkanie nr 16}

\noindent
\textbf{data}: 2014-06-02 \\
\textbf{czas}: 19.30-23.50

\vspace{5mm}
\noindent
\textbf{uczestnicy}:
\begin{itemize}
	\item Marcin Dzieżyc
	\item Piotr Kalinowski
	\item Adam Papros
	\item Mateusz Statkiewicz
	\item Marcin Swend
	\item Jacek Witkowski
\end{itemize}

\vspace{5mm}
\noindent
\textbf{cele}: \\
Finalizacja prac nad~projektem.
